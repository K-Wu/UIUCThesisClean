\section{Conclusion}
With the increasing adoption of GNNs in the machine learning community, GPUs have become essential to accelerate GNN training.
However, training GNNs on massive graphs that do not fit in GPU memory is still a challenging task.
Unlike conventional neural networks, mini-batching input samples
in GNNs requires complicated tasks such as traversing neighboring nodes and gathering their feature values.
While this process accounts for a significant portion of the training time, existing GNN implementations using popular deep neural network libraries such as PyTorch are limited to a CPU-centric approach for the entire data preparation step.
This ``all-in-CPU'' approach negatively impacts the overall GNN training performance as it over-utilizes CPU resources and hinders GPU acceleration of GNN training.
To overcome such limitations, we introduce PyTorch-Direct, which enables a GPU-centric data-accessing paradigm for GNN training.
In PyTorch-Direct, GPUs can efficiently access complicated data structures in host memory directly without CPU intervention.
Our microbenchmark and end-to-end GNN training results show that PyTorch-Direct reduces data transfer time by 47.1\% on average and speeds up GNN training by up to 1.6$\times$.
To minimize programmer effort, we introduce a new ``unified tensor'' type along with necessary changes to the PyTorch memory allocator, dispatch logic, and placement rules. 
As a result, users need to change at most two lines of their PyTorch GNN training code for each tensor object to take advantage of PyTorch-Direct.
